\newpage
\phantomsection
\addcontentsline{toc}{chapter}{Glossary}
\chapter*{Glossary}
\markboth {Glossary}{Glossary}

Of course there are plenty of glossaries out there! One (not too serious) example is the online MT glossary of Kevin Knight \myfootnote{Machine Translation Glossary (Kevin Knight): \url{http://www.isi.edu/natural-language/people/dvl.html}} in which MT itself is defined as 
\begin{quote}
techniques for allowing construction workers and architects from all over the world to communicate better with each other so they can get back to work on that really tall tower. 
\end{quote}

\begin{description}
 \item[accuracy] A basic score for evaluating automatic \textbf{annotation tools} such as \textbf{parsers} or \textbf{part-of-speech taggers}. It is equal to the number of \textbf{tokens} correctly tagged, divided by the total number of tokens. [\ldots]. (See \textbf{precision and recall}.)

\item[clitic] A morpheme that has the syntactic characteristics of a word, but is phonologically and lexically bound to another word, for example \textit{n't} in the word \textit{hasn't}. Possessive forms can also be clitics, e.g. The dog\textit{'s} dinner. When \textbf{part-of-speech tagging} is carried out on a corpus, clitics are often separated from the word they are joined to.

\end{description}

